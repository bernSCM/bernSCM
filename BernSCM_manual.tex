\documentclass[11pt]{article}
\usepackage[utf8]{inputenc}
\usepackage[T1]{fontenc}
\usepackage{fixltx2e}
\usepackage{graphicx}
\usepackage{longtable}
\usepackage{float}
\usepackage{wrapfig}
\usepackage{rotating}
\usepackage[normalem]{ulem}
\usepackage{amsmath}
\usepackage{textcomp}
\usepackage{marvosym}
\usepackage{wasysym}
\usepackage{amssymb}
\usepackage{hyperref}
\tolerance=1000
\author{Kuno Strassmann}
\date{\today}
\title{Bern Simple Climate Model (BernSCM) manual}

\begin{document}

\maketitle

\noindent
The model as described in this manual is designed for a Unix/Linux shell environment. Model code is written in Fortran90 and set up for compiling with gfortran or pgf90 (tested with gfortran 5.4.1 and pgf90 7.2-3). Required additional tools are GNU make and sed for compiling, and (optionally) \mbox{xmgrace} for checking results.

\section*{Getting started}
\label{sec-1}

To generate executables and run example simulations, execute the script 
\begin{verbatim}
run_examples.sh
\end{verbatim}
To plot the example simulations, execute the scripts
\begin{verbatim}
xmgrace_pulse.sh
xmgrace_c4mip.sh
\end{verbatim}

The usage is explained in detail below.

\section*{1. Compiling}
\label{sec-2}

Within the directory where this readme file resides, do

\begin{verbatim}
make
\end{verbatim}

to compile with the default compiler gfortran, or alternatively

\begin{verbatim}
make profile=pgf90
\end{verbatim}

to compile with the pgf90 compiler (requires gnu make, and gfortran or pgf90).
This will produce the executable bernSCM.


\section*{2. Running a simulation}
\label{sec-3}

When bernSCM is executed, it will interactively query some parameters for the simulation (climate sensitivity, C cycle sensitivities to temperature/CO2, forcing file identifier (scenario), and an additional identifier for the simulation). This information can be entered by hand, but will typically be fed to the model by a "runfile", a text file containing one input parameter per line.

\begin{verbatim}
bernSCM < runfiles/run_mysimulation
\end{verbatim}

The contents of the runfile are, for example
\begin{verbatim}
2.5      #climate sensitivity
.true.   #temperature dependent
.true.   #CO2 dependent
c4mip_a2 #simulation ID
test     #Optional additional identifier
\end{verbatim}


\section*{3. Input data: the forcing file}
\label{sec-4}

The forcing file contains the boundary conditions or forcing of the simulation. 

Forcing files are located in the directory "forcing", and follow the naming convention
\begin{verbatim}
forcing_<scenario>.dat
\end{verbatim}
where <scenario> is an identifier entered as interactive input (see 2.).

Forcing files are text files containing a table with the following data columns: 
\begin{itemize}
\item simulation time (Time/yr)
\item global mean surface temperature deviation from preindustrial (glob\_temp\_dev/K)
\item non-CO2 radiative forcing (RF\_nonC/Wm-2)
\item radiative forcing for budget closure (budget\_RF/Wm-2)
\item atmospheric CO2 (co2\_atm/ppm)
\item fossil/anthropogenic CO2 emissions (fossil\_em/GtC/yr)
\item C sink flux for budget closure (budget\_sink/GtC/yr).
\end{itemize}

Important specifications:
\begin{itemize}
\item All forcing variables must be present IN THE GIVEN ORDER (header is not read)
\item The model is set up to run from preindustrial equilibrium, thus initial time must be preindustrial.
\item Data may be arbitrary spaced in time and will be interpolated.
\item Each record refers to a POINT in time.
\item Some columns will contain the "missing value" flag -9999.9999 (NA). NAs indicate variables to be determined by budget closure.
\item At the initial record, ALL variables must be defined (not NA).
\item All physical quantities may be determined by budget closure, but for RF and C emissions/uptake, dedicated budget variables must be used (budget\_RF, budget\_sink). Budget variables must be set to zero when not used for closure.
\item Variables used for budget closure may change along the simulation (e.g., different over historic and future period). When the budget variable changes, e.g. from to budget\_sink to co2\_atm, there must be a record with both these variables defined (not NA). This is because each record represents a point in time.
\item The following budget cases as indicated by NAs are possible:
\end{itemize}

\begin{center}
\begin{tabular}{lllll}
Budget & glob\_temp\_dev & budget\_RF & co2\_atm & budget\_sink\\
CT & NA & 0 & NA & 0\\
ET & NA & 0 & prescribed & NA\\
ER & prescribed & NA & prescribed & NA\\
CE & prescribed & prescribed & NA & NA\\
\end{tabular}
\end{center}

Legend:
\begin{description}
\item[{CT}] given CO2 emissions and non-CO2 RF, solve for atmospheric CO2 and temperature
\item[{ET}] given CO2 concentrations, emissions and non-CO2 RF, solve for residual CO2 uptake and temperature
\item[{ER}] given CO2 concentrations, emissions, non-CO2 RF, and temperature, solve for residual CO2 uptake and RF
\item[{CE}] given CO2 emissions, non-CO2 RF, and temperature, solve for atmospheric CO2 and CO2 uptake
\end{description}

For determining non-CO2 RF and allowable emissions, the entries budget\_RF and budget\_sink can be used, while setting the columns RF\_nonC and fossil\_em to 0.

\section*{4. Numerical solution}
\label{sec-5}

The code supports timesteps ranging from subyear to decadal. The larger timesteps require more complicated numerical schemes for a stable solution. Therefore the timestep is set at the preprocessing stage, by setting the preprocessor flag deltat in the file src/control.inc, for example:

\begin{verbatim}
#define deltat 1d0
\end{verbatim}

To set the appropriate numerical scheme, the file may contain the preprocessor flags shown in the following table:
\begin{center}
\begin{tabular}{llrlr}
numerical scheme & deltat (yr) & implicitO & implicitL & linear\\
\hline
Euler forward, simplest & \textasciitilde{}0.1 & 0 & 0 & 0\\
Implicit step(a) & \textasciitilde{}1 & 1 & (1) & 0\\
Implicit step(b) & \textasciitilde{}10 & 1 & (1) & 1\\
\end{tabular}
\end{center}

(a) For stability, implicit treatment is essential for ocean C, optional for land C.
(b) Linear discretization is needed for accuracy with large timesteps.


\section*{5. Model versions}
\label{sec-6}

The land/ocean pulse-response functions are specified in src/parLand*.inc, src/parOcean*.inc.
Similarly, NPP parametrizations are defined in src/parNPP*.inc, src/npp*.inc.

Different model components are available, including for the ocean HILDA (parOceanHILDA.inc), Bern2D (parOceanB2D.inc), and for the land the Bern 4box model (parNPP\_4box.inc, npp\_4box.finc), and the HRBM substitute model (parLandHRBM.inc, parNPP\_HRBM.inc, npp\_HRBM.finc).

When compiling, generic parameter files will be read in, thus to include any component, link or copy the corresponding file, e.g.:

\begin{verbatim}
parLandHRBM.inc -> parLand.inc
\end{verbatim}

Additional model components may be added in the same way, using the existing files as a template.

\subsection*{Note on land models:}
\label{sec-6-1}

\begin{itemize}
\item For temperature-dependent land models (like HRBM), the preprocessor flag
\end{itemize}

\begin{verbatim}
#define LandTdep 1
\end{verbatim}

in src/control.inc MUST be set to 1! Otherwise T-dependence will not be compiled in, leading to erroneus results without warning! For non-T-dependent models (4box), the flag must be removed or set to 0. 
\begin{itemize}
\item NPP and pulse response from different parent models may be combined if so desired (e.g., 4box NPP with HRBM biosphere).
\end{itemize}


\subsection*{Note on preprocessed code:}
\label{sec-6-2}

\begin{itemize}
\item The source code (src/*.F files) is not very readable because it accomodates a range of numerical schemes for different purposes. Practical applications will often use a specific version exclusively (e.g. with a given timestep), which correspond to preprocessed code produced when compiling (src/*.for files), using flags set in the file src/control.inc. Thus it is recommended that the more readable preprocessed code be used for the implementation of a specific model version.
\end{itemize}


\section*{6. Output control}
\label{sec-7}

Internally, the model defines state variables (CO2, Temp, etc.) at time points, while C fluxes are defined in between at mid-timestep.
Fluxes may be interpolated to time points with the preprocessor option dointerpol, which is set at the top of file src/bernSCM-output.F. Interpolation causes a negligible error in integrals. It is enabled per default (dointerpol=1), and can be switched off (dointerpol=0) if desired. 


\section*{7. Varia}
\label{sec-8}

\begin{itemize}
\item The preprocessor flag \#tequil (in src/control.inc) may be used to treat boxes with small turnover times as equilibrated. All timescales below tequil (yr) will be equilibrated, thus a value of 0, or missing flag, implies no equilibration. As equilibration does not strongly affect performance and may reduce accuracy, its use is not recommended.
\end{itemize}
% Emacs 24.5.1 (Org mode 8.2.10)
\end{document}
